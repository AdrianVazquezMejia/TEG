Las redes WiFi malladas se presentan como una buena opci�n para la implementaci�n de nuevos sistemas de automatizaci�n, control y/o monitoreo debido a su r�pida instalaci�n, bajos costos y una confiabilidad cada vez mayor. 

El \mcu posee una amplia cantidad de librer�as y perif�ricos, as� como potencia de procesamiento para el desarrollo de aplicaciones que requieran una alto rendimiento, rapidez de procesamiento y multitarea. Adem�s de un precio accesible comparado con otros que posee funcionalidades parecidas. El desarrollo de programas es estructurado, con un paradigma de programaci�n basado en el sistema operativo en tiempo real. El fabricante brinda  una amplia cantidad de ejemplos para una primera aproximaci�n. 

La capacidades WiFi del ESP-32 poseen caracter�sticas similares a los adaptadores de red convencionales en �rea de silicio muy peque�a. Lo que le da a este microcontrolador beneficios adicionales en conectividad e interoperatividad, en el auge de las redes WiFi y el IoT.

La librer�as de funcionalidades \espn es lo suficientemente flexible para implementar arquitecturas y protocolos de red WiFi, sin gastar demasiados recursos en configuraci�n en la pila TCP/IP. Sin embargo, posee una velocidad de transmisi�n m�xima hasta diez menor que las redes WiFi convencionales, por otro lado no posee los beneficios de chequeo de error de TCP, y la vinculaci�n entre puntos no es permanente, lo que se traduce en enlaces susceptibles a una gran perdida de datos. As� mismo, la carga m�xima permitida es de 250 bytes, por lo que reduce las posibilidades de \espn a aplicaciones que transmitan paquetes livianos, y protocolos de tramas cortas.  

La implementaci�n de una red de enrutamiento din�mico posee pocos beneficios en la red implementada, pues \mb ya posee un sistema de excepciones y es costoso determinar si las fallas son en la red o en los equipos \mb. As� mismo, la red se orienta a ambientes donde la posici�n de los nodos no variar�a.

La red \espn y el protocolo \mb se acoplan satisfactoriamente, debido a la bajas tasas de transmisi�n,  tramas ajustadas a la capacidad de red y tiempo entre solicitudes. Por otro lado, el principal inconveniente es la ausencia de  control de congesti�n, y si un nodo posee numerosos enlaces la tasa de p�rdida de paquetes puede incrementarse significativamente. 