Una exhaustiva revisi�n bibliogr�fica demostr� que  existe gran cantidad de documentaci�n actual respecto a las redes WiFi malladas, a su implementaci�n en distintas �reas y sobre distintos hardware, sin embargo, se identific� como aspecto resaltante  que cada una ajusta su funcionamiento al �rea de aplicaci�n, lo que fue de ayuda para restringir el alcance de la red dise�ada.


 Se logr� implementar una red mallada usando el microcontrolador ESP32, haciendo uso de las librer�as de  funciones brindadas por el fabricante que agilizan el desarrollo de aplicaciones. Adem�s, el uso del sistema operativo en tiempo real permiti� que se implementaran programas en un esquema sencillo y organizado.  

Con el uso del chip MAX-3485 y el programa de manejo de UART en el microcontrolador ESP32, se logr� establecer una comunicaci�n serial entre el microcontrolador y un equipo Modbus. Dicha comunicaci�n estuvo soportada sobre una interfaz RS485 half-duplex de dos conductores. 

La red \espn se ajust� a  el protocolo \mb  satisfactoriamente, debido a la bajas tasas de transmisi�n, tama�o tramas  dentro de la capacidad de red y tiempo entre solicitudes. Por lo que obtuvo una red que soport�  la transmisi�n del protocolo Modbus inalabricamente,  sin que representar� una degradaci�n en la eficacia de comunicaci�n respect� a linea serial, dentro de ciertos l�mites.

Se obtuvo un dise�o del hardware de un nodo que funcione en la red dise�ada. Dicho dise�o contempl� el uso del microcontrolador ESP32, el chip MAX3485, la alimentaci�n y la interfaz f�sica de la conexiones. 


Finalmente, se ejecutaron pruebas variando par�metros de comunicaci�n Modbus, donde se logr� observar aquellas caracter�sticas cuya modificaci�n altera en mayor grado la eficacia de la red para la transmisi�n de datos, como lo es la topolog�a. 


