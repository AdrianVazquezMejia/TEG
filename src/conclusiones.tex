El ESP32 permite implementar topolog�as de red inal�mbricas  con  distintas arquitecturas, debido a que posee una librer�a de funciones brindadas por el fabricante que aguilizan el desarrollo de aplicaciones.Adem�s, el uso del sistema operativo en tiempo real permite que se implementen programas en un esquema sencillo y organizado.  

Existe gran cantidad de documentaci�n actual respecto a las redes WiFi Malladas y a su implementaci�n en distintas �reas y sobre distintos hardware, sin embargo, cada una ajusta su funcionamiento al �rea de aplicaci�n. La mayor cantidad de informaci�n relacionada se encuentra en Internet, debido a que son art�culos recientes.



La red \espn y el protocolo \mb se acoplan satisfactoriamente, debido a la bajas tasas de transmisi�n,  tramas ajustadas a la capacidad de red y tiempo entre solicitudes. Por otro lado, el principal inconveniente es la ausencia de  control de congesti�n, y si un nodo posee numerosos enlaces la tasa de p�rdida de paquetes puede incrementarse significativamente. 