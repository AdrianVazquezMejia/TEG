Implementar un modo de compuerta en TCP/IP en los nodos para abrir la posibilidad de acceder a otras redes.

Implementar un control centralizado de gesti�n de datos para disminuir la perdida de paquetes en nodos congestionados.

Crear un software con interfaz gr�fica que permita aplicar configuraciones a los nodos y la de manera r�pida, usando lenguajes como Java, Python, etc.

En cuanto al enrutamiento se recomienda realizar estudios para la implementaci�n de un gestor de rutas de red que use algoritmos del camino corto, de acuerto a las conexiones establecidas por el usuario, tomando como base el algoritmo de Dijsktra o el algoritmo de Prim.

Tambi�n pueden hacerse estudios de rendimiento la propagaci�n de las tramas broadcast usando algoritmos heur�sticos constructivos, observado este proceso como el problema del vendedor viajero (TPS). Esto para la aplicaci�n de actualizaciones de configuraciones comunes en los  nodos que componen la red o los equipos Modbus involucrados.

