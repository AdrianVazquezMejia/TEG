Se recomienda ampliar la capacidades de comunicaci�n  la red implementando un modo de compuerta en TCP/IP en en el nodo maestro, para abrir la posibilidad de acceder remotamente a esta, incluso Internet, creando una caracter�stica relacionada al Internet de la cosas en el �rea Industrial (IIoT).

Dada la relevancia de la topolog�a en el rendimiento de la red, debido a la congesti�n de informaci�n en los nodos con m�ltiples conexiones, se propone  implementar un control centralizado de gesti�n de datos  y de balanceamiento de red para disminuir la p�rdida de paquetes en nodos congestionados y que  establezca estructuras de red mas eficaces, de forma din�mica.


En cuanto al enrutamiento se recomienda realizar estudios para la implementaci�n de un gestor de rutas de red que use algoritmos del camino corto, de acuerto a las conexiones establecidas por el usuario, tomando como base el algoritmo de Dijsktra o el algoritmo de Prim. Suponiendo en cada caso que el peso de cada enlace en la intensidad de se�al recibida. 

Tambi�n proponen hacer estudios de rendimiento la propagaci�n de las tramas broadcast usando algoritmos heur�sticos constructivos, observado este proceso como el problema del vendedor viajero (TPS). Esto para la aplicaci�n de actualizaciones de configuraciones comunes en los  nodos que componen la red o los equipos Modbus involucrados.

