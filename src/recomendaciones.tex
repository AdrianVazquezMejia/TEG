Se recomienda implementar un enrutamiento din�mico en la red, para observar los beneficios respecto a el enrutamiento est�tico. 

Se propone aplicar una rutina de b�squeda de esclavos en cada nodo, as� cada nodo detectar�a los esclavos que posee conectados, para disminuir el tiempo de configuraci�n de la red.

Dise�ar una interfaz gr�fica para configurar la red,  para as� disminuir el tiempo y los errores  de establecimiento de la red.

Se propone implementar la configuraci�n de canal \wf  y la llaves locales mediante \mb, para poseer mayor extensibilidad de red.

Es recomendable  extender la configuraci�n para el soporte de m�s opciones de \mb serial, como el modo ANSII.

En el hardware del nodo, se deber�a agregar la interfaz mec�nica de RJ-45 muy popular en la industria. As� como puertos de entrada salida para otras posibles funciones de monitoreo.

Implementar el modo de maestro \mb en los nodos para que exista la posibilidad de que el nodo maestro funcione como puerta de enlace a otras redes, por ejemplo \mb TCP/IP.