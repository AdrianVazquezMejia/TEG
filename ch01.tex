
\section*{Planteamiento del problema}

La creciente popularidad de las redes inal�mbricas en casi todos los sectores ha hecho que las infraestructuras asociadas, dispositivos y protocolos se vean en la necesidad de mejorar constantemente para manejar la creciente cantidad de usuarios de manera segura y eficiente. As� mismo, las redes WiFi centralizadas est�n limitadas por la capacidad del punto de acceso, en el �mbito de cobertura y  cantidad de clientes\citep{range} , restricciones que se pudiesen superar con las redes WiFi malladas.

Actualmente  la redes malladas est�n en una etapa de desarrollo, por lo que no existe un est�ndar s�lido para la implementaci�n de toda la red, lo que causa reservas en las industrias, especialmente debido a la vulnerabilidad de los datos, la confiabilidad en ambientes electromagn�ticamente ruidosos y el rendimiento \citep{planteam}. Es por eso que una  red mallada WiFi con comunicaci�n basada en el protocolo Modbus  podr�a representar una soluci�n a este problema.

\section*{Justificaci�n}

Se plantea una red WiFi lo cual representar�a una reducci�n de costos en la implementaci�n de un sistema de control y adquisici�n de datos, dado que se necesitan menos conductores. Adem�s, se explorar� el campo popular hoy en d�a de las redes WiFi y su rendimiento como red mallada. Dicha red representa una alternativa a superar la limitaciones de n�mero de clientes y �rea de cobertura presentes en las redes WiFi centralizadas, y m�s a�n, supone una soluci�n a llegar a lugares lejanos de un nodo central sin la necesidad de agregar puntos de acceso adicionales.

La constituci�n de la red mallada se elaborar� basados en comunicaci�n WiFi a trav�s de microcontroladores ESP32, que poseen caracter�sticas de tama�o, potencia y costos aunado a las  ventajas principales de las redes mallada de cobertura y conectividad. Tambi�n se sustentar� la transmisi�n de informaci�n en el protocolo Modbus debido a su confiabilidad todos los sectores aplicables, especialmente el industrial.

\section*{Alcance y limitaciones}

La red se compondr� de al menos cuatro nodos, donde cada nodo tendr� conexi�n con al menos otro nodo usando red WiFi y  estar�n constituidos por microcontroladores ESP32  con m�dulo WiFi integrado y el programa asociado a la red. Los nodos deben estar apropiadamente alimentados, cuyo dise�o no forma parte del proyecto.

El programa se dise�ar� para que se logre transportar la informaci�n bajo el protocolo Modbus  por la red inal�mbrica, considerando que solo en un nodo est� conectado el maestro. As� mismo, algunos de los nodos restantes poseer�n esclavos. Cabe resaltar que las unidades que generan los datos del protocolo Modbus (maestro y esclavos) no forman parte de la red a dise�ar, ya que se asume que se recibe informaci�n en los  nodos sin tener en cuenta mayor detalle de su origen.