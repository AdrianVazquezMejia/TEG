\section{Dise�o de la red mallada}
 
 Cada red mallada se form� como para que la informaci�n del protocolo MODBUS proveniente de esclavos o maestro viaje por ella sin que represente ninguna diferencia respecto a una linea serial, es decir, para los elementos MODBUS es transparente la red. As�, la red se puede instalar para equipos que funcionen sobre la linea serial sin modificaci�n alguna.
 
 El enrutamiento de la red se lleva a cabo a partir del esclavo al que este interrogando el maestro. El maestro esta conectado serialmente a un nodo, este al recibir la informaci�n v�a serial identifica el esclavo y en consecuencia env�a la trama MODBUS a un nodo espec�fico que se encuentra el camino hacia el esclavo en cuesti�n.  Un nodo al recibir la trama inalambricamente identifica el esclavo en la trama y verifica si debe reenviar la trama a otro nodo, o en su defecto transmitirlo serialmente ya que posee el esclavo MODBUS en dicha interfaz. Luego de generada la respuesta esta es analogamente llevada hasta el maestro.
 
 La red es experta, es decir, conoce de antemano en que nodo se encuentra cada esclavo, lo cual es usado para enrutar los mensajes.
 
%agregar esquema 
 \subsection{Caracter�sticas de los nodos}
 Se realiz� el programa para que todos los nodos tuviesen el mismo c�digo, sin importar el rol que posea en la red (maestro, intermedio o final). Se llama nodo maestro a el que posee el maestro conectado, nodo intermedio aquel cuya funcionalidad solo en reenviar datos y nodo final a el que posee esclavos Modbus. Sin embargo, los nodos finales tambi�n pueden reenviar datos.
 
 Los nodos en s� soportan el protocolo Modbus para ser interrogados y configurados. Por lo que los se reservaron direcciones para  los nodos, en este caso los identificadores desde 101 a 255.  
 
 Cada nodo se le agreg� la posibilidad de ofrecer el servicio de configuraci�n de su identificador Modbus,la tasa de baudios de la interfaz RS-485, y la tabla de enrutamiento. Esto a trav�s del acceso a los registros de retenci�n. Adem�s de la funcionalidad de reinicio y reseteo de fabrica(hardware). %xxx Agregar que otras funcionalidades.
 Las direcciones de los registros con sus funcionalidades se expresan en la tabla \ref{Tabla 1: Espec HR}.
 
 \begin{table}[H]
  \centering
\begin{tabular}{llr}
 \toprule
Direcci�n & Tipo de registro & Descripci�n  \\
\midrule
01 & Registros de rentenci�n & Identificador \\
02 & Registros de rentenci�n    &  Tasa de baudios  \\
256-512 & Registros de rentenci�n  & Tabla de enrutamiento\\
0 & Bobina &Reinicio \\
\bottomrule
\end{tabular}
\caption{Tablas primarias de los modelos de datos Modbus}\label{Tabla 1: Espec HR}
\end{table}

M�s concretamente, los nodos de la red son esclavos adicionales de la red Modbus.
 %Nodos modbus, configuracion, serial'wifin %limitaciones
 
 \subsection{Caracter�sticas de enrutamiento}
 
 Bien sea que los datos sean recibidos serial o inalambricamente,  las decisiones siguientes se toman en base a tabla de enrutamiento.
 
 La tabla de enrutamiento consisten en un arreglo de 256 casillas, donde la posici�n est� asociada a el esclavo Modbus y el valor en la casilla se relaciona con la ubicaci�n de dicho esclavo. Cabe resaltar que la ubicaci�n a la que se refiere la tabla de enrutamiento no es la posci�n del esclavo en la red, en su lugar es el siguiente nodo a reenviar los datos para llegar al esclavo en cuesti�n, es decir, los nodos solo conocen un salto hacia adelante y un salto hacia atr�s.

La tabla de enrutamiento se llena  mediante los registros de retenci�n, a partir de la direcci�n 256, la cual se asocia al esclavo 1. Esta debe ser determinada por el usuario que conoce la posici�n de los nodos en el �rea a implementar. 
 
Cuando se recibe un trama, se extrae de la trama el identificador del esclavo, y surgen dos casos: corresponde a un n�mero diferente de cero � es cero. Si es cero, entonces el nodo no tiene ruta a ese nodo, en caso contrario el n�mero puede corrponder al indentificador del nodo en cuesti�n o a otro nodo al cual se reenviar�n la trama. En caso de que sea el nodo en cuesti�n entonces ha llegado la trama ha llegado a su destino. 

Un esquema m�s detallado del enrutamiento se puede apreciar en el esquema de la figura  \ref{Fig 1: ULM enrutamiento}.
 
 %tablas de enrutamiento 
