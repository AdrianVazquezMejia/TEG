\section{Dise�o de una red mallada basada en el microcontrolador ESP32}

La red mallada se dise�o para una sola red Modbus, por lo que, solo puede existir un maestro y una cantidad de XXX esclavos (Citacion required). Teniendo en consideraci�n lo anterior, se defini� la topolog�a de la red mallada; las rutas de comunicaci�n se definieron seg�n el esclavo que el maestro Mosbus est� interrogando. As�, aunque las rutas son est�ticas, el enrutamiento de cada nodo varia seg�n una tabla fija de enrutamiento determinada. 

Por otro lado, existen otras caracter�sticas  de comunicaci�n como lo es el est�ndar del bus serial y la caracter�sticas del mismo. Ya que se usa el est�ndar RS-485, quedaba por definir la tasa de transmisi�n. Esta puede variar, por lo que el nodo es configurable y compatible para las tasas de Baudios siguientes: 

Entonces, quedando diferenciadas las dos etapas (la inalambrica y la serial) y adem�s, se consideraron los dos sentidos de la informaci�n (desde y hacia el esclavo), para establecer una l�gica para el funcionamiento del nodo. 

Cada nodo contemplaron dos sentidos, el primero es si se le introduce la informaci�n por el bus serial entonces esta debe transmitirse inalambricamente; y si recibe informaci�n inalambricamente entonces debe pasarla a su bus serial o bien debe trasnmitirla a otro nodo.

Cada  nodo debe ser configurable respecto a su identificador, tabla de enrutamiento y  tasa de baudios de la interfaz serial, para as� adaptarlos a los entornos de comunicaci�n industriales. Sabiendo que se transmite el protocolo Modbus entoinces se consider� conveniente considerar a los nodos como esclavos en el sistema. Se reservaron identificadores para los nodos, lo que deja menos cantidad de identificadores de esclavos disponibles para la red en la que se implemente la red mallada.

La comunicaci�n entre los nodos es unicast para as� aprovechar las ventajas que esto implica como lo son encriptaci�n, respuesta de agradecimiento y ahorro de (sobrecarga). Para tener mayor cobertura, los esclavos pueden estar a mas de un nodo intermedio del nodo que posee el maestro y la red se dise�o para poseyera la capacidad de reenviar dicha informaci�n  convenientemente.

Definidas estar caracter�sticas de funcionamiento, se investigaron las librer�as que el microcontrolador ESP32 pose�a para hacer tales funciones. Se encontraros dos tipos de redes WiFi descritos por Espressif: mesh y espnow. Para elegir cual usar se tom� en cuenta la flexibilidad, descentralizaci�n, seguridad y alcance; pues bien son las cualidades que entraban en concordancia para la elaboraci�n de la arquitectura de red descrita anteriormente.

\section{Implementar el m�dulo del programa para el manejo del protocolo Modbus en el bus RS-485.}

Como se  trabaj� en en microcontrolador ESP32, los programas elaborados son administrados por un sistema operativo en tiempo real (RTOS). Al mismo tiempo, el dise�o de los programas contempl� todas las caracter�sticas que un RTOS implica. Esto se traduce en un paradigma de programaci�n basado en tareas, eventos, colas, etc.
