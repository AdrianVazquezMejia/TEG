\section{Dise�o de una red mallada basada en el microcontrolador ESP32}

La red mallada se dise�o para una sola red Modbus, por lo que, solo puede existir un maestro y una cantidad de XXX esclavos (Citacion required). Teniendo en consideraci�n lo anterior, se defini� la topolog�a de la red mallada; las rutas de comunicaci�n se definieron seg�n el esclavo que el maestro Mosbus est� interrogando. As�, aunque las rutas son est�ticas, el enrutamiento de cada nodo varia seg�n una tabla fija de enrutamiento determinada. 

Por otro lado, existen otras caracter�sticas  de comunicaci�n como lo es el est�ndar del bus serial y la caracter�sticas del mismo. Ya que se usa el est�ndar RS-485, quedaba por definir la tasa de transmisi�n. Esta puede variar, por lo que el nodo es configurable y compatible para las tasas de Baudios siguientes: 

Entonces, quedando diferenciadas las dos etapas (la inalambrica y la serial) y adem�s, se consideraron los dos sentidos de la informaci�n (desde y hacia el esclavo), para establecer una l�gica para el funcionamiento del nodo. 

Cada nodo contemplaron dos sentidos, el primero es si se le introduce la informaci�n por el bus serial entonces esta debe transmitirse inalambricamente; y si recibe informaci�n inalambricamente entonces debe pasarla a su bus serial o bien debe trasnmitirla a otro nodo.

Cada  nodo debe ser configurable, para as� adaptarlos a los entornos, y ya que se transmite el protocolo Modbus se consideran los nodos como esclavos. Es decir, se reservaron identificadores para los nodos para as� configurar tanto el enrutamiento como la tasa de trasnmision serial.
